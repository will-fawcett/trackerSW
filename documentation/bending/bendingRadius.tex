\documentclass[a4paper, 12pt]{article}

\usepackage{amsmath}
\usepackage{amssymb}

\begin{document}

Charged particles will bend in a magnetic field. 
For the purposes of a tracking detector, it is important to know how much a particle's trajectory will change. 
Shown below is some of the mathematics used to calculate bending radii in certain cases. 

For the ATLAS detector, the barrel tracker is cylindrical around the beam-pipe. 
Charged particles emerge from the centre of this cylinder, and 

We consider the lorentz force law
\begin{equation}
  \mathbf{F} = q( \mathbf{E} + \mathbf{v} \times \mathbf{B})
\end{equation}



\end{document}
