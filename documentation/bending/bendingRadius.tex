\documentclass[a4paper, 12pt]{article}

\usepackage{amsmath}
\usepackage{amssymb}

\begin{document}

\section{Introduction}
Charged particles will bend in a magnetic field. 
For the purposes of a tracking detector, such as the one used in ATLAS, it is important to know how much a particle's trajectory will change as it traverses a tracker. 
Shown below is some of the mathematics used to calculate bending radii in certain cases, and some plots illustrating this.  

\subsection{Coordinate system}
The coordinate system is shared with the ATLAS experiment. 
The $z$ direction is along the beam pipe, the positive $y$ direction vertically upwards and the $x$ direction towards the centre of the LHC ring. 

\section{Bending radius calculation}
We consider the lorentz force law
\begin{equation}
  \mathbf{F} = q( \mathbf{E} + \mathbf{v} \times \mathbf{B})
\end{equation}
For the ATLAS detector, the barrel tracker is cylindrical around the beam-pipe. 
The magnetic field is generated by a solenoid which surrounds the cylindrical tracker, and as a result points only in the $z$ direction, so 
$\mathbf{B} = B \mathbf{\hat{z}}$. 
There is no electric field. 
The trajectory of a charged particle will therefore only be modified in the $x$--$y$ plane, and we can therefore collapse the problem to that plane. 
To simplify further, we can assume that the initial velocity of the charged particle is along the $x$ direction, $\mathbf{v} = v \mathbf{\hat{x}}$
The lorentz force law then collapses to
\begin{equation}
  F  = qvB
\end{equation}
With the force acting in the $y$-direction. 




\end{document}
