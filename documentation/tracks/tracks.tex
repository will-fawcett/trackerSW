\documentclass[a4paper, 12pt]{article}

\usepackage{amsmath}
\usepackage{amssymb}
\usepackage{graphicx}
\usepackage{xspace}
\usepackage{cleveref}
\usepackage{booktabs}
\usepackage[parfill]{parskip}
\usepackage{booktabs}


\newcommand{\pt}{\ensuremath{p_{\mathrm{T}}}\xspace}

\begin{document}

\section{Tracks}
The myTrack class is used to take a vector of hit objects and calculate the track parameters. 
At current, two track parameter extraction methods are used: beamline constraint and no-beamline constraint. 
First, some definitions.
\begin{description}
  \item[A fake track]
    A track is labelled as fake if any one of the hits belonging to the track do not come from the same particle.

  \item[Track reconstruction efficiency] 
    The ATLAS convention defines this as the fraction of truth particles matched to a reconstructed track (after the track passes the relevant quality criteria).
    Here, we used the following definition
    \begin{equation}
      \epsilon = \frac{\mathrm{Number~of~tracks~matched}}{\mathrm{Number~of~particles~in~outermost~layer}}
    \end{equation}
    The number of particles in the outermost layer is used as a proxy for the number of true particles. 
    The number of matched tracks is equivalent to the number of truth particles matched to a track. 
    This efficiency can be calculated as a function of the reconstructed tracks $\eta$ or \pt, or of the truth particles $\eta$ or \pt. 

  \item[Fake rate]
    The fake rate is defined as 
    \begin{equation}
      R_f = \frac{\mathrm{Numer~of~fake~tracks}}{\mathrm{Numer~of~reconstructed~tracks}}.
    \end{equation}
    This can be defined as a per event quantity, or as a function of $\eta$ or $\pt$. 

\end{description}

\subsection{Beamline constraint}
All tracks are assumed to have only three hits, one for each layer of the triplet.






\end{document}
