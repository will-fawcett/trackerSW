\documentclass[a4paper, 12pt]{article}

\usepackage{amsmath}
\usepackage{amssymb}
\usepackage{graphicx}
\usepackage{xspace}
\usepackage{cleveref}
\usepackage{booktabs}
\usepackage[parfill]{parskip}
\usepackage{booktabs}


\newcommand{\pt}{\ensuremath{p_{\mathrm{T}}}\xspace}

\begin{document}

\section{Tracks}
The myTrack class is used to take a vector of hit objects and calculate the track parameters. 
At current, two track parameter extraction methods are used: beamline constraint and no-beamline constraint. 
First, some definitions.
\begin{description}
  \item[A fake track]
    A track is labelled as fake if any one of the hits belonging to the track do not come from the same particle.

  \item[Track reconstruction efficiency] 
    The ATLAS convention defines this as the fraction of truth particles matched to a reconstructed track (after the track passes the relevant quality criteria).
    Here, we used the following definition
    \begin{equation}
      \epsilon = \frac{\mathrm{Number~of~tracks~matched}}{\mathrm{Number~of~particles~in~outermost~layer}}
    \end{equation}
    The number of particles in the outermost layer is used as a proxy for the number of true particles. 
    The number of matched tracks is equivalent to the number of truth particles matched to a track. 
    This efficiency can be calculated as a function of the reconstructed tracks $\eta$ or \pt, or of the truth particles $\eta$ or \pt. 

  \item[Fake rate]
    The fake rate is defined as 
    \begin{equation}
      R_f = \frac{\mathrm{Numer~of~fake~tracks}}{\mathrm{Numer~of~reconstructed~tracks}}.
    \end{equation}
    This can be defined as a per event quantity, or as a function of $\eta$ or $\pt$. 

\end{description}

\subsection{Beamline constraint}
All tracks are assumed to have only three hits, one for each layer of the triplet.
To extract the track parameters, it is assumed that the particle originated form the beamline, i.e. $(0, 0, z_0)$ in $(x, y, z)$. 
Only the hits in the outermost and innermost layer are used, with coordinates $(x_1, y_1, z_1)$ and $(x_3, y_3, z_3)$ respectively, giving a total of three coordinates.
By definition $d_0=0$. 


To extract the radius of curvature the centre of the circle connecting the three coordinates is calculated in the $x$--$y$ plane as follows:
\begin{align}
  a = \frac{y_3 r_{01}^2 - y_1 r_{03}^2}{2(y_3x_1 - y_1x_3)} \\
  b = \frac{x_3 r_{01}^2 - x_1 r_{03}^2}{2(y_3x_1 - y_1x_3)} 
\end{align}
where $r_{0i} = \sqrt{x^2_i + y^2_i}$ and $(a, b)$ are the coordinates of the centre of the circle. 
Since the circle touches the origin, the radius of curvature is then $R=\sqrt{a^2 + b^2}$ from which the \pt can be extracted using $\pt[\mathrm{GeV}] = 1.199 \times R[\mathrm{m}]$, which holds for a
magnetic field of 4\,T.

As an aside, one can further simplify the expression for $R$, consider $R^2 = a^2+b^2$ and let $C_{13} = x_1 y_3 - y_1 x_3$, 
\begin{align}
  a^2 + b^2 & = \frac{(y_3 r_{01}^2 - y_1 r_{03}^2)^2 + (x_3 r_{01}^2 - x_1 r_3^2)^2}{4C_{13}^2} \\ 
  & = \frac{r^4_{01} y_3^2 + r_{03}^4 y_1^2 - 2 r^2_{01}r^2_{03}y_1 y_3    + r^4_{01} x_3^2 + r_{03}^4 x_1^2 - 2 r^2_{01}r^2_{03}x_1 x_3      }{4C_{13}^2} \\ 
  & = \frac{r_{01}^4 (x_3^2 + y_3^2) + r_{03}^4 (x_1^2 + y_1^2) -2r^2_{01}r^2_{03}(x_1 x_3 + y_1 y_3)}{4 C_{13}^2} \\ 
  & = \frac{r^2_{01}r^2_{03}(r^2_{01} + r^2_{03} -2(x_1 x_3 - y_1 y_3))}{4C_{13}^2} \\ 
\end{align}
Noticing that $r^2_{13} = (x_3 - x_1)^2 + (y_3 - y_1)^2 = r_{01}^2 + r_{03}^2 - 2(x_1 x_3 + y_1y_3)$, we have
\begin{equation}
  R = \sqrt{a^2 + b^2} = \frac{r_{01} r_{03} r_{13}}{2C_{13}}
\end{equation} 


The $\phi$ parameter is calculated as the angle made by the tangent to the circle at the origin. 
The angle is given by $\phi = \tan^{-1}( -b/a)$. 

In the longitudinal plane $z_0$ and $\eta$ can also be extracted. 
We follow the methodology of A. Sch\"oening.
First the bending angles at the first and third layer positions are calculated
\begin{align}
  \Phi_1 & = 2 \sin^{-1} \left( \frac{C_{13}}{r_{13} r_{03} }\right) \\
  \Phi_3 & = 2 \sin^{-1} \left( \frac{C_{13}}{r_{13} r_{01} }\right)
\end{align}
where $C_{13} = x_1 y_3 - y_1 x_3$.
Then the corresponding arc lengths are calculated
\begin{align}
  s_1 & = R \Phi_1 \\
  s_3 & = R \Phi_3 
\end{align}
From which the track parameters are extracted as follows:
\begin{align}
  z_0 & = z_1 - s_1 \frac{z_3 - z_1}{ s_3 - s_1} \\
  \theta & = \tan^{-1} \left( \frac{s_3 - s_1}{ z_3 - z_1} \right)
\end{align}

\subsection{No beamline constraints}
Here, the above assumption that the track originated from $(0, 0, z_0)$ is removed, and all three hit points are used to extract the track parameters. 

\section{Results} 
Track parameters are calculated with the beamline constraint by default.



\end{document}
